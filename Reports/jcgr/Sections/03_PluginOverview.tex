\section{Overview of the Plugin}
\label{03}

The plugin Jakob and I have created allows for basic tracking of objects and events in a game created in Unity. The gathered data is stored in an XML file and can be used inside Unity to create a basic heat map for the game. 

The events that gets tracked are grouped by what object they belong to, which makes it easy to generate a heat map for a certain type of event on a single object (such as the player dying, jumping, firing a bullet, enemies spawning, etc.) or a combination the individual events. It is also possible to access all events for a given object, a given event for all objects or all events for all objects.

The plugin can, by default, track some default Unity events such as the position of an object. Furthermore, one can track custom events, allowing the developers to gather virtually any kind of data they want. The heat map generation will take the custom events into account, and allow generation of heat maps of the custom events.

As seen in \refFig{H_01}, the heat map overlay generated by the plugin is of a decent quality, and certainly good enough to learn something about the players' behaviour from. As one can see, the player just walked around until they got to something interesting, at which point they stayed for a little while, investigating it.

\insertPicture{0.8}{Hiraeth_01}{A heat map showing player movement in the game Hiraeth.}{H_01}

\subsection{Parts of the Plugin}
\label{03_01}

The plugin consists of four folders, each containing components needed for tracking and visualizing data. 

\subsubsection*{Editor}
\label{03_01_01}

The \textbf{editor} folder contains two scripts, \texttt{HM\_TrackerEditor} and \texttt{HM\_GenerateHeatMapEditor}. Both scripts "support" other scripts (\texttt{HM\_Tracker} and \texttt{HM\_GenerateHeatMap}), and their primary job is to modify the Unity inspector to make the scripts they support easier to use.

\subsubsection*{Plugins}
\label{03_01_02}

The \textbf{plugins} folder contain two scripts that are used internally in the heat mapping, \texttt{Colorx} and \texttt{HSBColor}. These two scripts are used to control the color of the markers that are used to represent the position of events in the game. Both of them have been written by Jonathan Czeck, Graveck Interactive 2006.

\subsubsection*{Prefabs}
\label{03_01_03}

The \textbf{prefabs} folder contains the two objects that will be used to generate and show the heat map. The \texttt{HM\_HeatMap} is used to generating the heat map from the gathered data, while the \texttt{HM\_HeatMarker} shows each position generated from the data.

\subsubsection*{Scripts}
\label{03_01_04}

The \textbf{scripts} folder contains six scripts that are part of the heat mapping process.

\begin{my_itemize}

	\item The \texttt{HM\_ControlObjectScript} script is put on a dummy object that is generated when the game start. It ensures that only one folder is created for the data gathering session, and it takes care of combining the data for the different objects when the game ends.

	\item The \texttt{HM\_Event} script is very basic. It holds the type of event and the position at which it happened and can write this data to an XML file.

	\item The \texttt{HM\_EventTypes} script holds the default type of events the plugin can track.

	\item The \texttt{HM\_GenerateHeatMap} script takes care of generating the heat map based on the data gathered, and it creates and places the markers in the world. The data is uses is chosen through the Unity inspector options.

	\item The \texttt{HM\_HeatMarkerScript} script is placed on the \texttt{HM\_HeatMarker} objects, and is used to change the color on the marker.

	\item The \texttt{HM\_Tracker} script is placed on the object one wants to gather data for. It will track the object's position by default and can be set up to track some other events as well. It will save this data to an XML file for the session that is in progress (determined by the \texttt{HM\_ControlObjectScript} script).

\end{my_itemize}

\subsection{???}
\label{03_02}
