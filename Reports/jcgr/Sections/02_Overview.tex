\section{The Project}
\label{02}

When it comes to video games, there is a lot of data available from when people are playing and all of it can be tracked. Data that involves the positions of objects, where entities died (and what they died to), where certain events were performed, etc. can also be very interesting for the developers. It can be used for statistics, influencing balance, finding bottlenecks and figure out what is interesting for the players.

While the data is important, displaying it in an easy-to-understand manner is at least as important. Using an FPS\footnote{First Person Shooter} game as an example, showing what weapons the players prefer to use overall, their accuracy with weapons, which enemies are killed with which weapon, etc. can be visualized properly with a table. It works because the data is numbers and the numbers are not necessarily related to where in the world the even takes place.

When it comes to positioning of any kind in a game, a table is not a good choice. A table is useful for saying how many times an event happened, but it is not very helpful when it comes to stating \textit{where} the event happened. "Event e has happened at location x, y, z so and so many times" gives some information, but where in the game is x, y, z?

\insertPicture{0.45}{HalfLife_HeatMap}{A heat map representing deaths in a level in Half-Life 2}{HL_2_HeatMap}

This is where heat maps have their place. Using colors, they can show where - and how many times - an event has happened in the game. Heat maps are maps of the different places of the game with colored spots on. The colors of the spots range from an icy blue to a dark red. The darker the color is, the more times a certain event has happened there.

Looking at \refFig{HL_2_HeatMap} on page \pageref{fig:HL_2_HeatMap}, one place is bright red, which means that players have died a lot at that position. This can either indicate that the place is too hard, that the players have missed something that can heighten their survival, or that the place is working as an intended difficult area. Either way, the heat map gives the developers useful information, which is easy to understand and work with.

\subsection{Goal of the Project}
\label{02_01}

As both Jakob and I are rather interested in game balance and mechanics, we decided we wanted to make a tool that could assist in generating heat maps. After talking it through, we decided to create the tool for Unity, for a couple of reasons.

Unity works on a set of rules. Every object in the game world has a position and every object has access to methods that are the same for all objects (such as the Update() method, which is called every frame on every object). This means we can assume that every object can do certain things, and base our tool on that.

Another nice things is that plugins in Unity are actually just collections of scripts and prefabricated objects. As such, it is easy to export and import in another game. Copy the necessary scripts and objects, and you have a working heat map tool for another game.

Unity is also fairly easy to work with, when it comes to creating new tools for it. The API used in Unity is solid, and gives access to basically anything one will need. The documentation and support for Unity is also extensive, making it easy to figure out what one can do and how to do it.

\subsection{Features}
\label{02_02}

Track events on all kinds of objects.

Gather data in XML files and combine to one file.

Chose a set of data (object/event) to display.

Work offline.

Ease of use. (Finding a tool that seems to cover all your needs, just to figure out that it takes ages to set up properly is never fun. In fact, it can turn people away from using the tool, even if may be what they are looking for. Therefore, one of our goals was to make the plugin easy to set up and easy to use.)