\section{How To Use}
\label{HowToUse}
The heatmapping plugin consists of two parts: tracking objects and generating the heatmap.

To access the plugin, the custom package must be imported. The package consists of 4 folders. \textit{Scripts/HM\_Tracker} and \textit{Prefabs/HM\_HeatMap} are the elements used by the developer to create heatmaps. The rest of the files should not be used.

\subsection{Tracking game objects}
\label{HowToUse_Tracking}
To track a game object, the \textit{HM\_Tracker} script must be attached to the object. Attaching the \textit{HM\_Tracker} to an object allows the user to change the tracking interval, save interval and which events that are toggled.
\\The tracker can record the position of the game object when the following events happen: Breadcrumb (object position at intervals), Awake, onDestroy, OnMouseUp and OnTriggerEnter.

INSERT SCREENSHOT OF TRACKER OPTIONS

The OnMouseUp event is only tracked on every call to Update on the game object. Assuming 30 calls to update per second (30 fps), the tracker will register a maximum of 30 mouse up events per second.

The tracking interval determines the minimum interval between every breadcrumb in seconds. This value cannot be less than one second\footnote{If the value was lower, it could become a memory hog and take too long to save event info.}.

The save interval is the minimum time between the positions being saved to file\footnote{Like the tracking interval, this value cannot be less than 1. We felt that saving event information more often than breadcrumbs can be tracked makes sense.}. 
If a game uses a lot of memory, a lower value here would mean less memory is used, but would also mean that the harddrive is accessed more often.

The tracked data is saved in a file named "HeatMapData/HeatMapData YYYY.MM.DD HH.MM.SS.xml"\footnote{Tracked data for individual game objects are saved in folder "HeatMapData/YYYY.MM.DD hh.mm.ss/HeatMapDataNameOfGameObject.xml".}.

Because we track this way, tracked object names must be unique. 

\subsubsection{Custom events}
There are two ways to track custom events. 

The first way is to create a separate script that calls the AddEvent method of the \textit{HM\_Tracker} component on the game object whenever the event happens. This is how we designed custom events to be tracked.

The second way is to modify the \textit{HM\_tracker} script, so that the event is tracked from inside the script itself.

\subsection{Generating the heat map}
\label{HowToUse_Generating}
A copy of the \textit{Heatmapping/Prefab/HM\_HeatMap} prefab should be part of the unity scene hierarchy. The prefab has a script attached that allows for generating heatmap visuals. The \textit{Heat Marker} variable of the script should not be changed. 

INSERT SCREENSHOT OF HEATMAP OPTIONS

\textit{Heat Marker Scale} determines the scale of the sphere surrounding a tracked event. \textit{Allowed Distance} determines how far away two positions can be from each other before they count as increased density.

\textit{Session Data} lets the user choose from different datasets in the "HeatMapData" folder. Currently, it is only possible to load data files from the "HeatMapData" folder.

\textit{Generate Heatmap}
