\section{How To Use}
\label{HowToUse}
The heatmapping plugin consists of two parts: tracking objects and generating the heatmap.

To access the plugin, the custom package must be imported. The package consists of 4 folders. Only the \textit{Scripts/HM\_Tracker} and \textit{Prefabs/HM\_HeatMap} should be accessed.

\subsection{Tracking game objects}
\label{HowToUse_Tracking}
To track a game object, the \textit{HM\_Tracker} script must be attached to the object. Attaching the \textit{HN\_Tracker} to an object allows the user to change the tracking interval, save interval and which events that are toggled.
\\The tracker can record the position of the game object when the following events happen: Breadcrumb (object position at intervals), Awake, onDestroy, OnMouseUp and OnTriggerEnter.

INSERT SCREENSHOT OF TRACKER OPTIONS

The OnMouseUp event is only tracked on every call to Update on the game object. Assuming 30 calls to update per second (30 fps), the tracker will register a maximum of 30 mouse up events per second.

The tracking interval determines the minimum interval between every breadcrumb in seconds. This value cannot be less than one second\footnote{If the value was lower, it could become a memory hog and take too long to save event info.}.

The save interval is the minimum time between the positions being saved to file\footnote{Like the tracking interval, this value cannot be less than 1. We felt that saving event information more often than breadcrumbs can be tracked makes sense.}. 
If a game uses a lot of memory, a lower value here would mean less memory is used, but would also mean that the harddrive is accessed more often.

\subsubsection{Custom events}

\subsection{Generating the heat map}
\label{HowToUse_Generating}
