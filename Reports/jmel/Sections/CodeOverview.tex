\section{Overview of the Plug-in}
\label{CO}
The plug-in consists of eight scripts (six behaviour scripts, two editor scripts), two prefabs and one external plug-in (consisting of two scripts). 

\subsection{Folder structure}
\label{CO_FStructure}
The plug-in contains four folders:
\begin{my_enumerate}
\item [\textit{Editor}] contains custom editor menues for the two of the scripts in the plug-in.
\item [\textit{Plugins}] contains two scripts from an external plug-in.
\item [\textit{Prefabs}] contains two prefab objects used in the plug-in.
\item [\textit{Scripts}] contains the scripts for the main logic of the plug-in.
\end{my_enumerate}

\subsection{Prefabs}
\label{CO_Prefabs}
The plug-in uses two prefabs to render heat maps. The \textit{HM\_HeatMarker} is used for marking event positions and the density/heat of these positions in the heat map. The \textit{HM\_HeatMap} is essentially an empty game object with the \textit{HM\_GenerateHeatMap} script attached.

\subsection{HM\_Tracker}
\label{CO_Tracker}
This script is attached to objects to track (see section \ref{HowToUse_Tracking}). By default, it tracks the position of the object at different intervals (breadcrumbs) and can be set up to track other events. 
\\The tracker script keeps track of both how often to track breadcrumbs and how often to save tracked events to file. Between every save, the tracker keeps a list of all occured events. This list is reset on every save to reduce the risk of memory problems. 
\\Every tracker scripts also contains its own individual writer used to generate the XML files used for the data sets.

\subsection{HM\_HeatMarkerScript}
\label{CO_HMS}
\textit{HM\_HeatMarkerScript} contains the logic for calculating the density of a heat marker based on the heat markers around it. Calculating the density of a heat marker is O(n) where n is equal to the amount of heat markers in the scene.

\subsection{HM\_GenerateHeatMap}
\label{CO_GHM}
\textit{HM\_GenerateHeatMap} is in charge of generating the heat markers for the heat map. It does this in four steps:
\subsubsection{Load Session Data}
When the Generate Heat Map button in the editor is clicked, the script loads the specified XML data file.
\subsubsection{Instantiate Heat Markers}
After the data set has been loaded, the script scans through the data set and instantiating heat markers based on the options selected in the editor (game object(s) and event type(s)) and sets their parent to the object containing the \textit[HM\_GenerateHeatMap} script (usually the \textit{HM\_HeatMap} prefab).
\subsubsection{Calculate Density}
\label{CO_GHM_CD}
After instantiating the heat markers, the script makes every heat marker calculate their density and figures out the highest density. This runs in O(n\textsuperscript{2}) because every heat marker must check every other heat marker to calculate the density.
\subsubsection{Set Color}
Once the density of all heat markers has been determined, the script calculates the relative density of every heat marker and generates a colour based on this relative density. This operation runs in O(n) time.

\subsection{HM\_ControlObjectScript}
\label{CO_COS}
The \textit{HM\_ControlObjectScript} makes sure that only one data folder is created for each tracked game session. Additionally, it ensures that the seperate data files for each tracked game object is collected into one data file at the end of the game session. At the start of the game session, a dummy object containing this script is automatically instantiated. 

\subsection{HM\_Event & HM\_EventTypes}
\label{CO_Event}
\textit{HM\_Event} represents an event in the game and contains the logic for how to print itself in an XML-document.

\textit{HM\_EventTypes} is a helper class that contains the names of the standard event types used in the plug-in.

\subsection{Editor scripts}
\label{CO_Editor}
The two editor scripts change the appearance of the \textit{HM\_GenerateHeatMap} and \textit{HM\_Tracker} by adding drop-down lists to select data files, toggle options for tracked events and buttons for generating and clearing the heat map.
\subsection{External plug-in}
\label{CO_PlugIn}
The external plug-in is only used for a smooth colour transition on based on the density. The plug-in is called \footnote{Source: ColourX package http://wiki.unity3d.com/index.php/Colorx}.
